\documentclass[12pt, a4paper]{NGPLB}
\usepackage{tikz}
\usepackage[most, minted]{tcolorbox}
\usepackage{minted}
\usepackage{etoolbox}
\usepackage{indentfirst}
\usepackage{geometry}
\geometry{top = 2cm, bottom = 2cm, left = 2 cm, right = 2cm}
\tcbset{fontupper=\small, breakable, skin=bicolor ,listing engine={minted} ,colback=gray!30!white, colbacklower=gray!5!white, colframe=gray!30!white}

\newtcblisting{myoutput}{fontupper=\small, breakable, skin=bicolor ,colback=gray!30!white, colbacklower=gray!5!white, colframe=gray!30!white listing engine={minted}, listing only, minted language = latex}

\newtcblisting{mylisting}{fontupper=\small, breakable, skin=bicolor ,colback=gray!30!white, colbacklower=gray!5!white, colframe=gray!30!white, listing engine={minted}, listing only, minted language = latex}


\linespread{1.35}
\title{中正開源社章程解析}
\author{Si manglam}
\pagenumbering{arabic}
\setlength{\parindent}{24pt}
\begin{document}
\maketitle
\chapter{前言}

首先,說到起草章程,我想第一步就是先好好地靜下來,想想自己為甚麼想要創這個社團?為什麼這件事情是重要的?最重要的,我對這件事情的想像是什麼?這些問題都應該在這份章程中得到答案,並且每一條規範後的邏輯都應被清楚地記錄下來。只有滿足以上條件時,這份章程才會是一本好讀、好懂並合理的章程。

本章程使用 \LaTeX\ 輔以 Literate Programming 的方式撰寫,一來是為了要符合本社創立之宗旨「推廣開源文化、落實開源精神」所以選用開源軟體 \LaTeX\ 撰寫(主要是因為發起人很喜歡 \LaTeX\ )。二來則是為了更好的紀錄起草章程時的思緒,所以選用 Literate Programming 將章程各部分切分,並詳細紀錄轉寫各條例寫時的考量。

另外,等等章程中出現的我皆是指周造麟,我怕未來如果我畢業後有人要改這個章程但不知道可不可以用我自稱,答案是可以的,但請記得標注這裡的我是誰,不然一率算在我頭上喔!

\chapter{想像}

正式開始寫章程之前,我覺得我需要先將我的想像寫出來,畢竟最重要的是要有夢想,並願意為夢想爆肝,人類社會的進步就是靠這一些願意為了夢想而爆肝的人所推動的,希望某天我也可以成為這樣的人。

\section{總覽}

我目前對整個社團的想像還算是一片空白,不過依舊是有基礎的雛形。首先是社團的基礎,我希望整個社團是以「開源精神」與「生活駭客」這兩樣事物為核心去設計社課的。

\begin{figure}
\begin{center}
\begin{tikzpicture}
\node {有待畫圖--太久沒畫都快忘了怎麼畫};
\end{tikzpicture}
\end{center}
\end{figure}

至於為什麼是這兩樣?首先,開源是開放原始碼的縮寫,但其可帶來的效益卻遠遠不止如此。原始碼是將程式以人類可讀方式紀錄的載體,而程式又是解決問題的工具,所以開源最核心的概念應為:「將自己解決問題的方式公布出去」。藉由將原始碼公佈出去的行為,我們可以讓他人了解自己解決問題的方式並從中互相學習,並讓整個社群的人的實力更上一層樓。而更廣闊的來說,就算公布的不是原始碼、不是程式或軟體,只要能達到同樣的效果,我想那些都應該是有效的實踐開源精神的方式。

再來,雖然一般人對駭客的想像是躲在房間內破壞網路安全,而不是一群熱愛折騰自己、想辦法改問題的人。雖說駭客精神與開源精神有一部分的重疊,可是駭客比較像癡迷於技術,而開源比較像是癡迷於人與人之間的互動、合作共好,雖說我也很喜歡研究技術,不過我認為我是比較傾向開源的人,技術的推進是離不開人與人之間的交流的。

所以本社會推廣駭客精神,但卻不會以駭客精神為本純粹是因為我的關係,如果未來想要改變方向也可,我是沒有意見,畢竟我沒有辦法預知未來時代的變化,也許未來的學弟妹對駭客精神比較有興趣。

而為什麼要推行「開源精神」與「生活駭客」,最簡單的原因是:「Why Not?」。如同 Linus (Linux 的創始人)的自傳名稱「Just For Fun.」為什麼好玩、自己喜歡不可以是個理由,而非要找一個冠冕堂皇的理由去說服自己呢?

但屈服於這個社會下,我還是得找一個聽起來合理的理由。理由是以下幾點:

\begin{itemize}
\item 跨領域性&內容廣闊性
\item 招生容易性
\item 意義性
\end{itemize}

首先先從第一點開始說起,我們可以推廣的主題不一定只有技術相關的內容,我們可將內容從單純技術擴展到更多的面相,可能是用技術解決問題、可能是技術帶來的社會問題,而並非僅限於研究技術與理論。且我希望這個社團的運行方向是以解決社會問題為主要目的,而技術則是在解決問題時才需要學習到的事務。

第二點是招生容易性,如果以「研究軟體工程」或「精進 Code 實作」抑或是「競賽程式研究」之類的宗旨創設,我想要進入社團時需要的心理建設會過高,導致大家都覺得這個社團需要非常強的另一個是我認為我也不是這方面很強的人,我只是一個喜歡開源文化的初心者,我無法到來如此專業的課程,且我認為技術再怎麼樣都會回歸到人身上,與其研究技術,不如研究、推廣「開源」這個由技術帶來的哲學。

第三點則是意義性,雖說我本人是一個很偏技術類型的人,但我總覺得開源不只應該包含技術,而應該著重人與人之間的交流。這件代表如果一昧的追求技術上的進步,反而可能會忽略開源文化中最重要的一塊「與他人交流」,而這正是對我來說「開源」最重要的核心意義之一。

另外,駭客精神也是被包含在意義性內。我認為駭客精神的非常值得推廣,努力將身活周遭的一切系統化且改進,這不是「活到老學到老」不然是什麼。世界總是在變的,如果可以像駭客一樣想必可以在日新月異的世界中活得很好吧!

基於以上考量,所以我才決定以這兩件事情為主軸。

\section{社課方向}

社課則會希望是像以下圖表一樣安排:

\begin{figure}[ht]
\begin{tikzpicture}

\end{tikzpicture}
\end{figure}

\chapter{創社宗旨與任務}

\section{創社宗旨}

本社設址在中正大學內(也不能設在其他地方)並以「推廣開源文化、落實開源精神」為創社宗旨。


\begin{mylisting}
\item 總則
\begin{enumerate}
\item 本社定名為「中正開源社」(以下簡稱本社)。
\item 本社設址於國立中正大學。
\item 本社以「推廣開源文化、落實開源精神」為創社宗旨創立。
\end{mylisting}


\section{創社任務}

接著是本社的任務,基本上在我的構想中與創設理念相差不遠,皆是以實施創社理念為主。目前本會的任務大多是我想做的事或各種天馬行空的想法,我先寫出來,之後再一條條解釋原因。


\begin{mylisting}
\item 本社之社務如下:
\begin{enumerate}
\item 推廣開源文化
\item 促進校園開源精神
\item 協助辦理開源相關學術活動
\item 協助實施開源相關計畫
\item 其他與本社創社宗旨有關業務
\end{enumerate}
\end{enumerate}
\end{mylisting}


第一二條應該不用我過多解釋,畢竟這個社團本來就是以推廣開源文化為宗旨創立的,如果社務中沒影推廣開源文化,那豈不是自己打自己的臉,也違背了整個創社宗旨。讓我們把目光放到第三與第四條,這兩條其實才是我對整個社團未來的想像「我希望中正開源社可以在南部辦個 conf」,雖說南部已有 MOPCON,可 MOPCON 不是以開源為主軸的活動,以開源為主軸的活動在台灣大概只有 COSCUP 而已,而 COSCUP 又在北部,所以南部的開源活動可以說是少之又少,所以才希望可以在南部辦 conf。

不過說是這麼說,我還是認為 conf 的內容需要好好的思考,畢竟目前有的 conf 幾乎把所有的面向都包含到了,不論是以學生為主體的 SITCON 還是以開源為主體的 COSCUP 都已經深耕許久,我們必須好好找到自己的定位,並找出讓自己獨特的使命,但把這件事情當成對未來的暢想應該不過分吧?目前是覺得如果可以辦一個性質類似 g0v 的 conf 不錯,但性質類似要到多類似?我們跟 g0v 的差別又是什麼?這些都是需要被好好思考的事情。

至於協助實施開源相關計畫,我想說如果有其他社群或類似性質的團體有相關計畫要執行,可是卻找不到經費或合作夥伴,我們可能可以提供一些協助?雖然我們應該也很窮,但我們是學校內的社團,可以讓他人借我們當橋樑利用學校資源?(這應該可以吧?)

\chapter{社員}

\section{社員入社、退社與除名}

首先是社員的定義,只要是具有本校學籍的大學生並認同且願履行本社創社宗旨都可加入本社,成為本社的一員。至於具體加入方式就是入社手續,需向秘書組報名、登記(這裡就不限制方法了,只要是可以達到以上兩點即可視為有效的入社手續。),並繳納會費才會被視為完成手續並成為社員。


\begin{mylisting}
\item 社員
\begin{enumerate}
\item 凡具有本校學籍者且認同本社宗旨的學生皆可透過辦理入社手續成為本社社員。
\end{mylisting}


目前的設想是期末舉辦社員大會,並選舉出各行政組別下學期的組長。而入社手續則是在下學期初由社長指示秘書組辦理,這樣就不會出現在期初社員都還沒確認下來的情況就辦社員大會的情況了。畢竟只有社員有選舉權,如果社員與非社員混在一起會很難辦。

接下來是退社,一樣是秘書組負責退社手續,具體來說要向秘書組提出申請(申請書 or 表單)並會依當下日期退費,退費標準會以剩餘社課的比例退還。


\begin{mylisting}
\item 凡是本社社員皆可在任何時候向秘書組提出退社申請,秘書組通過申請後,已繳退費會按剩餘比例返還。
\end{mylisting}


最後是除名,要好好想想在怎樣的情況下才會走到除名這個地步。目前想到的情況有以下幾點:


\begin{mylisting}
\item 本社成員若有任意一項以下行為,社長、副社長得將該名社員除名,已繳交會費不另行退費。
\begin{enumerate}
\item 違反本社行為準則且屢勸不聽者
\item 未盡本社社員之義務者
\end{enumerate}
\end{mylisting}


基本上第一條就像是 COSCUP 或 SITCON 中會出現的行為準則(Code of Conduct),他們會有這條的原因是為了最大程度的保證每個人在活動中都是開心的,而我們會有這條的原因也差不多,都是要盡我們所能的保護社員不受傷害。詳細的行為準會參考 SITCON 與 COSCUP 修訂出一版適合中正開源社的。

剩下第二條應該就不需要過多的解釋,未盡社員義務就無法享有本社社員之權利,自然也不可被認為是本社之社員,所以得由社長與副社長除名。至於如果是社長或副社長犯了這些條例,那就可以靠罷免案,並在罷免成功後由代理社長除名。

\section{社員權利與義務}

這部分大概是最難的吧,我們需要簡明的定義出社員與他們應享有的權利與該履行的義務,這部分最麻煩了,要好好地想出我們可以提供什麼,同時也要想想我們需要社員提供我們什麼?但我們還是可以從最基礎的基本權開始,再一步步的建構出社員的定義。


\begin{mylisting}
\item 本社社員均享有以下權利
\begin{enumerate}
\item 優先參加本社舉辦之各項活動。
\item 發言權、表決權、罷免權、提案權、監督權、選舉與被選舉權。
\item 擔任本社行政組別之成員
\end{enumerate}
\end{mylisting}


基本上以上就是在保障進入本社的社員的各項權利受到保護,不論是對社團未來的投票、提案、表決、發言、對行政團隊的監督、罷免以及參政權。當然繳了錢也會有些許的好處,可以優先參與本社舉辦的活動。雖說開源社辦的活動應切合開源精神,不限制活動參與者的身份,但這樣的話就沒有人想要講會費了,於是還是得給繳了會費的人一點優待。

再來就是義務,其實義務大概也就是寫一寫,不會特別嚴格的執行,因為義務已經是最低標準了,要是連最低標準都做不到,那為什麼他會想來我們社啊?


\begin{mylisting}
\item 本社社員應盡以下義務
\begin{enumerate}
\item 定期參與本社之社課
\item 繳交本社所收之社費
\item 遵守本社訂定之章程、行為準則
\end{enumerate}
\end{enumerate}
\end{mylisting}


目前我能想到的就這兩項,這兩項大概是對社員最低的限制了吧。

\chapter{組織與職權}

接下來是對整個社團架構與職權的描述。目前看中正的規定,感覺社團的架構比較像英國的政治體系,所以讓我們向英國借一點組織架構的想法。英國的政治體系是內閣制,是由議會中選出首相並由首相任命行政部長,首相須對議會負責、議會須對人民負責。那我們也照這個大綱下去建設社團架構。

社團大概也會像這樣以議會為最高權力機構,並選出首相、讓首相任命各行政部部長,而部員則採招募的方式。目前與英國最大的不同點大概就是本社採取直接民主,而非間接民主(因為兩者的人口量級差太大了)。


\begin{mylisting}
\item 組織與職權
\begin{enumerate}
\item 本社由全體社員共同組成。
\item 本社以社員大會為最高權力機構。本社之行政機構,於會員大會閉會期間代行其職權。
\item 本社置社長、副社長一名,下設秘書部、課程部、美宣部、技術部。各部置部長一人、部員若干人。
\end{enumerate}
\end{mylisting}


\section{社員大會}

接著是對社員大會的描述:


\begin{mylisting}
\subsection{社員大會}
\begin{enumerate}
\item 社員大會為本社最高權力機構,其職權如下:
\begin{enumerate}
\item 制定、修改本會章程。 
\item 監督本社各項業務之運作
\item 開會時具有質詢、提案、表決等各項權利
\item 議決本會相關重要事物
\end{enumerate}
\end{enumerate}
\end{mylisting}


就是以議會為範本寫出的。

\section{會長}

會長則希望是經選舉產生,可以連任,並在社員大會休會時有著最後決策權。社長是對整個社團影響最大的人,所以這部分要特別小心,要給社長足夠大的權力,但也不能給太多的權力。權力的分立與約束要做好。


\begin{mylisting}
\subsection{社長}
\begin{enumerate}
\item 本會置會長一人,任期一年,得連選連任。會長之職權如下:
\begin{enumerate}
\item 對外代表本社、對內縱整本社事務。
\item 得代表本社出席各項會議
\item 需向社員大會負責,定期報告社團活動並接受質詢。
\item 組織內閣幹部並依需要增減
\item 召開並主持本社各項業務會議
\end{enumerate}
\end{enumerate}
\end{mylisting}


這裡很簡單,社長對外代表整個社團,對內要對社員大會負責並要接受質詢,其餘時間要負責帶領團隊執行任務。社長權力的界定這樣就做好了,接下來的副社長就會與社長差不多,但會有更多的限制。

\section*{附錄一:章程檔案 Preamble}


\begin{mylisting}
\documentclass[10pt, a4paper]{cos}
\title{中正開源社章程}
\author{Si manglam}
\end{mylisting}


\section*{附錄二:更改日期紀錄}


\begin{mylisting}
\begin{flushright}
什麼時候更改
\end{flushright}
\end{mylisting}


\section*{附錄三:本章程結構概要}


\begin{mylisting}
<<preamble>>
\begin{document}
\maketitle
<<更改紀錄>>
\begin{enumerate}
<<總則>>
<<社員入社、退社與除名>>
<<社員權利與義務>>
<<組織與職權>>
\end{enumerate}
\end{document}
\end{mylisting}


\end{document}
